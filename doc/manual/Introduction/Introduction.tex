\chapter{Introduction}
\section{Motivation}
Magentix2 is an agent platform for open Multiagent Systems. Its main objective is to bring agent technology to real domains: business, industry, logistics, e-commerce, health-care, etc.

Magentix2 platform is proposed as a continuation of the first Magentix platform. The final goal is to extend the functionalities of Magentix, providing new services and tools to allow the secure and optimized management of open Multiagent Systems. Nowadays, Magentix2 provides support at three levels: 

\begin{itemize}
	\item Organization level, technologies and techniques related to agent societies.
  \item Interaction level, technologies and techniques related to communications between agents.
 \item Agent level, technologies and techniques related to individual agents (such as reasoning and learning).
\end{itemize}

Thus, Magentix2 platform uses technologies with the necessary capacity  to cope with the  dynamism of the system topology and with flexible interactions, which are both natural consequences of the distributed and autonomous nature of its components. In this sense, the platform has been extended in order to support flexible interaction protocols and conversations, indirect communication and interactions among agent organizations. Moreover, other important aspects cover by the Magentix2 project are the security issues.

 


\section{Manual Structure}

In the following chapters, how Magentix2 platform must be installed, configured and used for programming agents is explained. 

Specifically, chapter \ref{chap:quickstart} clarifies how Magentix2 can be fully installed in only one host in a quickly and easy way. Furthermore, it is also explained how to develop and to execute simple Magentix2 agents. 

Chapter  \ref{chap:programmingAgents} is about programming aspects in Magentix2. Specifically, it is possible to consult  the basic classes of agents that the platform provides and the main issues related with agent communication. 

The advanced conversational agents class is explained in chapter \ref{sec:CAgents}. These agents enable the participation on simultaneous  conversations based on interaction protocols.

In chapter \ref{chap:JasonAgents} how to program BDI agents in Magentix2 is explained. Thus, Magentix2 provides the classes jasonAgent and MagentixAgArch, which enables programming agents in AgentSpeak and run them in the Magentix2 platform. 

The argumentation API provided by Magentix2 is shown in chapter \ref{sec:introductionArgAgents}. This API allows agents to engage in argumentation dialogues to reach agreements about the best solution to a problem. 


Chapter \ref{sec:tracingService} explains how agents can share information in an indirect way by means of the tracing service provide by Magentix2.
 



Chapter \ref{chap:VirtualOrganizations} explains in detail the support for virtual organizations provided by the Magentix2 platform. In this way, this chapter gives details about how the \textsc{THOMAS} (Methods, Techniques and Tools for Open Multi-Agent Systems) framework has been integrated with Magentx2, and how Magentix2 agents can use it. 

Chapter \ref{chap:HTTPInterface} is about describes the HTTP service supplied by Magentix2 in order to facilitate the interaction between Magentix2 agents and the outside world.


In order to customize the Magentix2 platform installation or distribute it in diverse hosts, the chapter \ref{chap:PlatformAdministration} should be consulted. Concretely, this chapter is about administration and configuration aspects related with the different components of the platform: Apache Qpid, the implementation of AMQP (Advanced Message Queuing Protocol) used for agent communication; MySQL, the database server used to maintain persistent information about the virtual organizations manage by the platform; Apache Tomcat, which allows agents to access to and provide standard Java web services; and  advanced Magentix2 platform services, such as the services which allows the communications with external agents or with the \textsc{THOMAS} framework. %; and the security module, which provides key features regarding security, privacy, openness and interoperability.

 

   

  
